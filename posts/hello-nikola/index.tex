.. title: Hello Nikola !
.. slug: hello-nikola
.. date: 2017-10-25 13:02:54 UTC+08:00
.. tags: 
.. category: 
.. link: 
.. description: 
.. type: text

\begin{blockquote}從這裡開始編輯文章\end{blockquote}

類似於程式設計的 Hello World,在這裡也來一篇文章來當作開始使用 Nikola 寫作的起點吧。

以前認為讀書無用,實作能力至上,後來發現沒有理論基礎很多事情根本無法有效率的執行。
就在這樣的背景下最近又回大學讀書了,但這次,不再讀電腦相關科系,而是選擇了經濟。
看似「跳痛」,實則不然,經濟系的經濟觀念與數學訓練反而比資訊科系適合資料探勘、機器學習(畢竟會不會寫程式與有沒有學過程式設計無關,有讀過資工就懂了 XD),
很多課程也可跟程式能力相互輔助,唯一美中不足的是沒有離散數學課程吧。

因為在學校讀書的關係,需要做大量筆記,加上當 web 工程師當久了沒機會寫字,字跡變得非常凌亂。
所以開始嘗試使用 LaTeX 寫筆記,一用就愛上了,進而想在所有地方都使用 LaTeX,像是 Blog 文章。
這就是使用 Nikola 的動機。

Nikola 跟其他靜態網站產生器的差異在於 LaTeX 語法的支援程度,
Nikola 的 \href{https://plugins.getnikola.com/v7/latex_formula_renderer/}{latex_formula_renderer} 是直接使用電腦安裝好的 LaTeX 來 render,
還可以預先定義好 environment,讓撰寫文章時比較方便。之後要支援圖表,我想也不會太困難,畢盡已經有一個現成的類似的 Plugin 可以參考。

最後,希望這個網站真的能培養起我的寫作習慣,讓語言能力有所提升,還真怕因為國文避不了業啊 XD 